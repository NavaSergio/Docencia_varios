% Options for packages loaded elsewhere
% Options for packages loaded elsewhere
\PassOptionsToPackage{unicode}{hyperref}
\PassOptionsToPackage{hyphens}{url}
\PassOptionsToPackage{dvipsnames,svgnames,x11names}{xcolor}
%
\documentclass[
  letterpaper,
  DIV=11,
  numbers=noendperiod]{scrartcl}
\usepackage{xcolor}
\usepackage{amsmath,amssymb}
\setcounter{secnumdepth}{5}
\usepackage{iftex}
\ifPDFTeX
  \usepackage[T1]{fontenc}
  \usepackage[utf8]{inputenc}
  \usepackage{textcomp} % provide euro and other symbols
\else % if luatex or xetex
  \usepackage{unicode-math} % this also loads fontspec
  \defaultfontfeatures{Scale=MatchLowercase}
  \defaultfontfeatures[\rmfamily]{Ligatures=TeX,Scale=1}
\fi
\usepackage{lmodern}
\ifPDFTeX\else
  % xetex/luatex font selection
\fi
% Use upquote if available, for straight quotes in verbatim environments
\IfFileExists{upquote.sty}{\usepackage{upquote}}{}
\IfFileExists{microtype.sty}{% use microtype if available
  \usepackage[]{microtype}
  \UseMicrotypeSet[protrusion]{basicmath} % disable protrusion for tt fonts
}{}
\makeatletter
\@ifundefined{KOMAClassName}{% if non-KOMA class
  \IfFileExists{parskip.sty}{%
    \usepackage{parskip}
  }{% else
    \setlength{\parindent}{0pt}
    \setlength{\parskip}{6pt plus 2pt minus 1pt}}
}{% if KOMA class
  \KOMAoptions{parskip=half}}
\makeatother


\usepackage{longtable,booktabs,array}
\usepackage{calc} % for calculating minipage widths
\usepackage{caption}
% Make caption package work with longtable
\makeatletter
\def\fnum@table{\tablename~\thetable}
\makeatother
\usepackage{graphicx}
\makeatletter
\newsavebox\pandoc@box
\newcommand*\pandocbounded[1]{% scales image to fit in text height/width
  \sbox\pandoc@box{#1}%
  \Gscale@div\@tempa{\textheight}{\dimexpr\ht\pandoc@box+\dp\pandoc@box\relax}%
  \Gscale@div\@tempb{\linewidth}{\wd\pandoc@box}%
  \ifdim\@tempb\p@<\@tempa\p@\let\@tempa\@tempb\fi% select the smaller of both
  \ifdim\@tempa\p@<\p@\scalebox{\@tempa}{\usebox\pandoc@box}%
  \else\usebox{\pandoc@box}%
  \fi%
}
% Set default figure placement to htbp
\def\fps@figure{htbp}
\makeatother





\setlength{\emergencystretch}{3em} % prevent overfull lines

\providecommand{\tightlist}{%
  \setlength{\itemsep}{0pt}\setlength{\parskip}{0pt}}



 


\KOMAoption{captions}{tableheading}
\makeatletter
\@ifpackageloaded{caption}{}{\usepackage{caption}}
\AtBeginDocument{%
\ifdefined\contentsname
  \renewcommand*\contentsname{Table of contents}
\else
  \newcommand\contentsname{Table of contents}
\fi
\ifdefined\listfigurename
  \renewcommand*\listfigurename{List of Figures}
\else
  \newcommand\listfigurename{List of Figures}
\fi
\ifdefined\listtablename
  \renewcommand*\listtablename{List of Tables}
\else
  \newcommand\listtablename{List of Tables}
\fi
\ifdefined\figurename
  \renewcommand*\figurename{Figure}
\else
  \newcommand\figurename{Figure}
\fi
\ifdefined\tablename
  \renewcommand*\tablename{Table}
\else
  \newcommand\tablename{Table}
\fi
}
\@ifpackageloaded{float}{}{\usepackage{float}}
\floatstyle{ruled}
\@ifundefined{c@chapter}{\newfloat{codelisting}{h}{lop}}{\newfloat{codelisting}{h}{lop}[chapter]}
\floatname{codelisting}{Listing}
\newcommand*\listoflistings{\listof{codelisting}{List of Listings}}
\makeatother
\makeatletter
\makeatother
\makeatletter
\@ifpackageloaded{caption}{}{\usepackage{caption}}
\@ifpackageloaded{subcaption}{}{\usepackage{subcaption}}
\makeatother
\usepackage{bookmark}
\IfFileExists{xurl.sty}{\usepackage{xurl}}{} % add URL line breaks if available
\urlstyle{same}
\hypersetup{
  pdftitle={Planteamiento del Problema y Justificación},
  pdfauthor={Sergio M. Nava Muñoz},
  colorlinks=true,
  linkcolor={blue},
  filecolor={Maroon},
  citecolor={Blue},
  urlcolor={Blue},
  pdfcreator={LaTeX via pandoc}}


\title{Planteamiento del Problema y Justificación}
\author{Sergio M. Nava Muñoz}
\date{2025-08-20}
\begin{document}
\maketitle


\section{Introducción}\label{introducciuxf3n}

\begin{itemize}
\tightlist
\item
  Todo proyecto de investigación comienza con un problema claro y bien
  definido.
\item
  La justificación es clave para demostrar la relevancia y viabilidad de
  la investigación.
\item
  Exploraremos cómo se estructuran el planteamiento del problema y la
  justificación.
\end{itemize}

\begin{center}\rule{0.5\linewidth}{0.5pt}\end{center}

\subsection{\texorpdfstring{\textbf{Diferencias entre el Planteamiento
del Problema y la
Justificación}}{Diferencias entre el Planteamiento del Problema y la Justificación}}\label{diferencias-entre-el-planteamiento-del-problema-y-la-justificaciuxf3n}

\begin{longtable}[]{@{}
  >{\raggedright\arraybackslash}p{(\linewidth - 2\tabcolsep) * \real{0.4773}}
  >{\raggedright\arraybackslash}p{(\linewidth - 2\tabcolsep) * \real{0.5227}}@{}}
\toprule\noalign{}
\begin{minipage}[b]{\linewidth}\raggedright
\textbf{Sección}
\end{minipage} & \begin{minipage}[b]{\linewidth}\raggedright
\textbf{Propósito}
\end{minipage} \\
\midrule\noalign{}
\endhead
\bottomrule\noalign{}
\endlastfoot
\textbf{Planteamiento del Problema} & Define \textbf{qué problema} se va
a investigar y cuál es su contexto. Describe la situación actual y
delimita la pregunta de investigación. \\
\textbf{Justificación} & Explica \textbf{por qué es importante}
investigar ese problema y cuáles serán los beneficios del estudio. \\
\end{longtable}

\begin{center}\rule{0.5\linewidth}{0.5pt}\end{center}

\subsection{\texorpdfstring{\textbf{¿Dónde se ubican dentro de un
trabajo de
investigación?}}{¿Dónde se ubican dentro de un trabajo de investigación?}}\label{duxf3nde-se-ubican-dentro-de-un-trabajo-de-investigaciuxf3n}

\textbf{Ejemplo de orden correcto en un proyecto de investigación:}

\begin{enumerate}
\def\labelenumi{\arabic{enumi}.}
\tightlist
\item
  Introducción
\item
  \textbf{Planteamiento del problema}
\item
  \textbf{Justificación}
\item
  Objetivos de la investigación
\item
  Marco teórico
\item
  Metodología
\item
  Análisis de resultados
\item
  Discusión
\item
  Conclusión
\item
  Referencias
\item
  Apéndices
\item
  Anexos
\end{enumerate}

\section{Planteamiento del Problema}\label{planteamiento-del-problema}

\subsection{¿Qué es el Planteamiento del
Problema?}\label{quuxe9-es-el-planteamiento-del-problema}

El \textbf{planteamiento del problema} es la base de toda investigación.
Define el contexto, justifica su importancia y delimita el foco de
estudio. Un problema de investigación es una brecha en el conocimiento o
una situación que requiere ser analizada y comprendida.

\begin{center}\rule{0.5\linewidth}{0.5pt}\end{center}

\subsection{Elementos Clave del Planteamiento del
Problema}\label{elementos-clave-del-planteamiento-del-problema}

\begin{enumerate}
\def\labelenumi{\arabic{enumi}.}
\tightlist
\item
  \textbf{Descripción del problema}: Presentar la situación actual,
  resaltando la relevancia del problema.\\
\item
  \textbf{Antecedentes}: Estudios previos, estadísticas o datos que
  contextualicen la problemática.\\
\item
  \textbf{Consecuencias}: Explicar el impacto de no abordar el
  problema.\\
\item
  \textbf{Formulación del problema}: Pregunta de investigación clara y
  delimitada.\\
\item
  \textbf{Objetivos}: Especificar qué se busca lograr con la
  investigación.
\end{enumerate}

\begin{center}\rule{0.5\linewidth}{0.5pt}\end{center}

\subsection{\texorpdfstring{Ejemplo 1: \textbf{Salud Pública -
Enfermedades Respiratorias en la
CDMX}}{Ejemplo 1: Salud Pública - Enfermedades Respiratorias en la CDMX}}\label{ejemplo-1-salud-puxfablica---enfermedades-respiratorias-en-la-cdmx}

\subsubsection{\texorpdfstring{\textbf{1. Descripción del
Problema}}{1. Descripción del Problema}}\label{descripciuxf3n-del-problema}

La \textbf{contaminación del aire en la Ciudad de México (CDMX)} es un
problema de salud pública que ha sido vinculado con enfermedades
respiratorias como el asma y la Enfermedad Pulmonar Obstructiva Crónica
(EPOC). Estudios recientes han demostrado un incremento en el número de
hospitalizaciones debido a la exposición prolongada a contaminantes como
\textbf{PM2.5 y NO₂}.

\subsubsection{\texorpdfstring{\textbf{2.
Antecedentes}}{2. Antecedentes}}\label{antecedentes}

\begin{itemize}
\tightlist
\item
  Según la OMS, la contaminación atmosférica es responsable de más de 7
  millones de muertes prematuras al año.\\
\item
  En 2023, la CDMX registró \textbf{42 días de contingencia ambiental}
  debido a altos niveles de ozono.\\
\item
  Investigaciones previas han analizado los efectos de la contaminación,
  pero pocos estudios han modelado su relación con la incidencia de
  enfermedades en tiempo real.
\end{itemize}

\subsubsection{\texorpdfstring{\textbf{3.
Consecuencias}}{3. Consecuencias}}\label{consecuencias}

Si no se toman medidas para mitigar el problema, las tasas de
enfermedades respiratorias continuarán aumentando, impactando el sistema
de salud pública y reduciendo la calidad de vida de los habitantes.

\subsubsection{\texorpdfstring{\textbf{4. Formulación del
Problema}}{4. Formulación del Problema}}\label{formulaciuxf3n-del-problema}

\textbf{¿Cuál es la relación entre los niveles de contaminación del aire
y la incidencia de enfermedades respiratorias en la CDMX en los últimos
cinco años?}

o

Este estudio analizará la relación entre los niveles de contaminación
del aire y la incidencia de enfermedades respiratorias en la Ciudad de
México en los últimos cinco años, con el objetivo de identificar
patrones, evaluar su impacto en la salud pública y proponer estrategias
de mitigación.

\subsubsection{\texorpdfstring{\textbf{5.
Objetivos}}{5. Objetivos}}\label{objetivos}

\begin{itemize}
\tightlist
\item
  Determinar los niveles de contaminación del aire en la CDMX en los
  últimos cinco años.\\
\item
  Analizar la correlación entre los niveles de contaminantes y el número
  de hospitalizaciones por enfermedades respiratorias.\\
\item
  Desarrollar un modelo predictivo que ayude a las autoridades a tomar
  decisiones basadas en datos.
\end{itemize}

\begin{center}\rule{0.5\linewidth}{0.5pt}\end{center}

\subsection{\texorpdfstring{Ejemplo 2: \textbf{Educación - Desigualdad
en el Acceso a la Educación
Superior}}{Ejemplo 2: Educación - Desigualdad en el Acceso a la Educación Superior}}\label{ejemplo-2-educaciuxf3n---desigualdad-en-el-acceso-a-la-educaciuxf3n-superior}

\subsubsection{\texorpdfstring{\textbf{1. Descripción del
Problema}}{1. Descripción del Problema}}\label{descripciuxf3n-del-problema-1}

El acceso a la educación superior en América Latina sigue siendo
desigual. Factores como la condición socioeconómica, la ubicación
geográfica y la falta de recursos educativos limitan las oportunidades
para jóvenes en comunidades rurales y marginadas.

\subsubsection{\texorpdfstring{\textbf{2.
Antecedentes}}{2. Antecedentes}}\label{antecedentes-1}

\begin{itemize}
\tightlist
\item
  Un informe de la UNESCO indica que \textbf{solo el 36\% de los jóvenes
  de zonas rurales en Latinoamérica accede a la educación superior},
  comparado con el 66\% en zonas urbanas.\\
\item
  Investigaciones previas han identificado que los costos de transporte
  y la falta de conectividad digital son barreras clave para los
  estudiantes rurales.\\
\item
  A pesar de los esfuerzos gubernamentales, las tasas de abandono
  universitario en estas comunidades siguen siendo altas.
\end{itemize}

\subsubsection{\texorpdfstring{\textbf{3.
Consecuencias}}{3. Consecuencias}}\label{consecuencias-1}

La desigualdad en la educación perpetúa la pobreza y limita el
desarrollo económico de las comunidades rurales. Sin oportunidades de
educación superior, los jóvenes tienen menos posibilidades de acceder a
empleos bien remunerados.

\subsubsection{\texorpdfstring{\textbf{4. Formulación del
Problema}}{4. Formulación del Problema}}\label{formulaciuxf3n-del-problema-1}

\textbf{¿Cuáles son los principales factores que limitan el acceso a la
educación superior para estudiantes de comunidades rurales en América
Latina?}

o

Este estudio investigará los principales factores que limitan el acceso
a la educación superior para estudiantes de comunidades rurales en
América Latina, identificando barreras económicas, sociales y
tecnológicas que afectan su ingreso y permanencia en instituciones
educativas.

\subsubsection{\texorpdfstring{\textbf{5.
Objetivos}}{5. Objetivos}}\label{objetivos-1}

\begin{itemize}
\tightlist
\item
  Identificar las principales barreras económicas, sociales y
  tecnológicas que afectan el acceso a la educación superior en zonas
  rurales.\\
\item
  Evaluar la efectividad de programas gubernamentales y becas en la
  reducción de la brecha educativa.\\
\item
  Proponer estrategias para mejorar la equidad en el acceso a la
  educación superior.
\end{itemize}

\begin{center}\rule{0.5\linewidth}{0.5pt}\end{center}

\subsection{\texorpdfstring{Ejemplo 3: \textbf{Cambio Climático -
Impacto del Calentamiento Global en la Biodiversidad
Marina}}{Ejemplo 3: Cambio Climático - Impacto del Calentamiento Global en la Biodiversidad Marina}}\label{ejemplo-3-cambio-climuxe1tico---impacto-del-calentamiento-global-en-la-biodiversidad-marina}

\subsubsection{\texorpdfstring{\textbf{1. Descripción del
Problema}}{1. Descripción del Problema}}\label{descripciuxf3n-del-problema-2}

El \textbf{calentamiento global} está afectando los ecosistemas marinos
a nivel global. Se ha observado una reducción en la biodiversidad de
arrecifes de coral, una migración anómala de especies marinas y una
disminución en la productividad de los océanos.

\subsubsection{\texorpdfstring{\textbf{2.
Antecedentes}}{2. Antecedentes}}\label{antecedentes-2}

\begin{itemize}
\tightlist
\item
  Según la NOAA, la temperatura de los océanos ha aumentado en promedio
  \textbf{1.5°C en los últimos 50 años}.\\
\item
  Un estudio publicado en \emph{Nature} advierte que más del
  \textbf{60\% de las especies marinas han cambiado su patrón de
  migración debido al cambio climático}.\\
\item
  Las barreras de coral en Australia han perdido \textbf{más del 50\% de
  su cobertura} en las últimas tres décadas.
\end{itemize}

\subsubsection{\texorpdfstring{\textbf{3.
Consecuencias}}{3. Consecuencias}}\label{consecuencias-2}

Si esta tendencia continúa, se perderán ecosistemas clave que sustentan
la pesca y la seguridad alimentaria. Además, la disminución en la
biodiversidad podría desencadenar colapsos ecológicos en diversas
regiones del mundo.

\subsubsection{\texorpdfstring{\textbf{4. Formulación del
Problema}}{4. Formulación del Problema}}\label{formulaciuxf3n-del-problema-2}

\textbf{¿Cómo está afectando el aumento de la temperatura del océano a
la biodiversidad marina en las zonas tropicales?}

o

Este estudio analizará el impacto del aumento de la temperatura del
océano en la biodiversidad marina en las zonas tropicales, identificando
los cambios en la distribución de especies, la pérdida de ecosistemas y
las posibles estrategias de adaptación para mitigar sus efectos.

\subsubsection{\texorpdfstring{\textbf{5.
Objetivos}}{5. Objetivos}}\label{objetivos-2}

\begin{itemize}
\tightlist
\item
  Evaluar los cambios en la distribución de especies marinas debido al
  calentamiento del océano.\\
\item
  Identificar las especies más vulnerables a estos cambios.\\
\item
  Analizar posibles estrategias de conservación para mitigar el impacto
  del calentamiento global en la biodiversidad marina.
\end{itemize}

\begin{center}\rule{0.5\linewidth}{0.5pt}\end{center}

\subsection{\texorpdfstring{Ejemplo 4: \textbf{Ciencias de la
Computación - Seguridad en el Uso de Inteligencia Artificial en la
Ciberseguridad}}{Ejemplo 4: Ciencias de la Computación - Seguridad en el Uso de Inteligencia Artificial en la Ciberseguridad}}\label{ejemplo-4-ciencias-de-la-computaciuxf3n---seguridad-en-el-uso-de-inteligencia-artificial-en-la-ciberseguridad}

\subsubsection{\texorpdfstring{\textbf{1. Descripción del
Problema}}{1. Descripción del Problema}}\label{descripciuxf3n-del-problema-3}

La \textbf{inteligencia artificial (IA)} está revolucionando la
ciberseguridad, permitiendo detectar ataques cibernéticos en tiempo
real. Sin embargo, también se han identificado vulnerabilidades que
pueden ser explotadas por ciberdelincuentes.

\subsubsection{\texorpdfstring{\textbf{2.
Antecedentes}}{2. Antecedentes}}\label{antecedentes-3}

\begin{itemize}
\tightlist
\item
  El 80\% de los ataques cibernéticos en 2023 involucraron algún tipo de
  IA, según \emph{Cybersecurity Ventures}.\\
\item
  Investigaciones recientes han demostrado que los modelos de IA pueden
  ser engañados con técnicas de \emph{adversarial attacks}.\\
\item
  A pesar de los avances en ciberseguridad, no existen estándares
  unificados para el uso seguro de IA en la detección de amenazas.
\end{itemize}

\subsubsection{\texorpdfstring{\textbf{3.
Consecuencias}}{3. Consecuencias}}\label{consecuencias-3}

Si no se abordan estas vulnerabilidades, las herramientas de seguridad
basadas en IA podrían ser utilizadas para facilitar ataques en lugar de
prevenirlos.

\subsubsection{\texorpdfstring{\textbf{4. Formulación del
Problema}}{4. Formulación del Problema}}\label{formulaciuxf3n-del-problema-3}

\textbf{¿Cuáles son las principales vulnerabilidades de los sistemas de
ciberseguridad basados en inteligencia artificial y cómo pueden
mitigarse?}

o

Este estudio investigará las principales vulnerabilidades de los
sistemas de ciberseguridad basados en inteligencia artificial,
analizando los riesgos asociados y explorando estrategias de mitigación
para fortalecer la protección contra amenazas cibernéticas

\subsubsection{\texorpdfstring{\textbf{5.
Objetivos}}{5. Objetivos}}\label{objetivos-3}

\begin{itemize}
\tightlist
\item
  Identificar las principales amenazas que enfrentan los modelos de IA
  en ciberseguridad.\\
\item
  Analizar estrategias para fortalecer la seguridad en el uso de IA.\\
\item
  Proponer un marco de estándares para el desarrollo seguro de modelos
  de IA en ciberseguridad.
\end{itemize}

\section{Conclusión}\label{conclusiuxf3n}

\begin{itemize}
\tightlist
\item
  El \textbf{planteamiento del problema} debe ser claro, relevante y
  justificable.\\
\item
  La \textbf{problemática debe basarse en datos y antecedentes} para
  demostrar su importancia.\\
\item
  La \textbf{pregunta de investigación} debe ser específica y responder
  a un problema real.\\
\item
  Los \textbf{objetivos} guían la investigación y delimitan su alcance.
\end{itemize}

\section{\texorpdfstring{\textbf{Formas de Justificación en la
Investigación}}{Formas de Justificación en la Investigación}}\label{formas-de-justificaciuxf3n-en-la-investigaciuxf3n}

\subsection{\texorpdfstring{\textbf{Definición}}{Definición}}\label{definiciuxf3n}

La \textbf{justificación de una investigación} es un elemento
fundamental en la elaboración de proyectos de estudio, ya que permite
explicar \textbf{por qué y para qué} se realiza el trabajo. Su propósito
es demostrar la \textbf{relevancia, pertinencia y viabilidad} del
estudio, estableciendo los beneficios que se derivarán de sus
resultados.

\begin{center}\rule{0.5\linewidth}{0.5pt}\end{center}

\subsection{\texorpdfstring{\textbf{¿Por qué es importante la
Justificación?}}{¿Por qué es importante la Justificación?}}\label{por-quuxe9-es-importante-la-justificaciuxf3n}

\begin{itemize}
\tightlist
\item
  Define la \textbf{importancia y utilidad} del estudio.\\
\item
  Permite \textbf{orientar la investigación} y su aplicabilidad.\\
\item
  Facilita el acceso a \textbf{financiamiento y apoyo institucional}.\\
\item
  Responde a preguntas clave:

  \begin{itemize}
  \tightlist
  \item
    \textbf{¿Por qué es importante este estudio?}\\
  \item
    \textbf{¿Qué beneficios traerá?}\\
  \item
    \textbf{¿Cómo contribuye a la generación de conocimiento?}
  \end{itemize}
\end{itemize}

\section{\texorpdfstring{\textbf{Tipos de Justificación en la
Investigación}}{Tipos de Justificación en la Investigación}}\label{tipos-de-justificaciuxf3n-en-la-investigaciuxf3n}

\begin{enumerate}
\def\labelenumi{\arabic{enumi}.}
\tightlist
\item
  \textbf{Falta de conocimiento}: Se argumenta la ausencia de estudios
  previos.
\item
  \textbf{Importancia del tema}: Se enfatiza la relevancia del fenómeno
  investigado.
\item
  \textbf{Aportes de los hallazgos}: Se destacan las posibles
  aplicaciones o implicaciones del estudio.
\item
  \textbf{Vacío metodológico}: Se justifica la investigación por el uso
  de nuevos enfoques o técnicas.
\item
  \textbf{Solución de un problema}: Se busca resolver una cuestión
  práctica o teórica.
\item
  \textbf{Corroboración de una teoría}: Se valida empíricamente un marco
  teórico existente.
\item
  \textbf{Justificación basada en criterios científicos}: Se argumenta
  con base en criterios como la \textbf{conveniencia, relevancia social,
  implicaciones prácticas y valor teórico}.
\end{enumerate}

\begin{center}\rule{0.5\linewidth}{0.5pt}\end{center}

\subsection{\texorpdfstring{\textbf{1. Justificación por Falta de
Conocimiento}}{1. Justificación por Falta de Conocimiento}}\label{justificaciuxf3n-por-falta-de-conocimiento}

Se fundamenta en la ausencia de estudios previos o en el conocimiento
limitado sobre un tema.

\textbf{Ejemplo:}\\
\emph{``No existen estudios detallados sobre el impacto del teletrabajo
en la salud mental en Latinoamérica, por lo que esta investigación
llenará ese vacío.''}

\textbf{Importancia:}\\
- Contribuye a generar \textbf{nuevo conocimiento}.\\
- Identifica \textbf{vacíos teóricos} en la literatura académica.

\begin{center}\rule{0.5\linewidth}{0.5pt}\end{center}

\subsection{\texorpdfstring{\textbf{2. Justificación por Importancia del
Tema}}{2. Justificación por Importancia del Tema}}\label{justificaciuxf3n-por-importancia-del-tema}

Argumenta que el estudio es relevante por su impacto en la sociedad, el
medio ambiente o la economía.

\textbf{Ejemplo:}\\
\emph{``El cambio climático es una de las principales amenazas para la
biodiversidad, por lo que este estudio analizará cómo afecta la
migración de especies marinas.''}

\textbf{Importancia:}\\
- Relacionado con \textbf{problemas actuales} de gran impacto.\\
- Puede influir en \textbf{decisiones políticas y sociales}.

\begin{center}\rule{0.5\linewidth}{0.5pt}\end{center}

\subsection{\texorpdfstring{\textbf{3. Justificación por Aportes de los
Hallazgos}}{3. Justificación por Aportes de los Hallazgos}}\label{justificaciuxf3n-por-aportes-de-los-hallazgos}

Explica cómo los resultados pueden \textbf{aplicarse} en el mundo real.

\textbf{Ejemplo:}\\
\emph{``El desarrollo de materiales biodegradables podría reducir la
contaminación plástica en un 60\%.''}

\textbf{Importancia:}\\
- Favorece la \textbf{innovación y el desarrollo tecnológico}.\\
- Impacta sectores como la \textbf{salud, la educación y la tecnología}.

\begin{center}\rule{0.5\linewidth}{0.5pt}\end{center}

\subsection{\texorpdfstring{\textbf{4. Justificación por Vacío
Metodológico}}{4. Justificación por Vacío Metodológico}}\label{justificaciuxf3n-por-vacuxedo-metodoluxf3gico}

Se basa en la necesidad de desarrollar o mejorar métodos de
investigación.

\textbf{Ejemplo:}\\
\emph{``Los modelos actuales de predicción de terremotos no consideran
la actividad volcánica. Este estudio propondrá un nuevo enfoque
combinado.''}

\textbf{Importancia:}\\
- Introduce \textbf{nuevas técnicas y enfoques} en la investigación.\\
- Mejora la \textbf{precisión y fiabilidad} de estudios previos.

\begin{center}\rule{0.5\linewidth}{0.5pt}\end{center}

\subsection{\texorpdfstring{\textbf{5. Justificación por Solución de un
Problema}}{5. Justificación por Solución de un Problema}}\label{justificaciuxf3n-por-soluciuxf3n-de-un-problema}

Se enfoca en la resolución de una \textbf{problemática concreta} en la
sociedad o en la ciencia.

\textbf{Ejemplo:}\\
\emph{``La escasez de agua en zonas áridas es un problema crítico. Este
estudio evaluará tecnologías de recolección de agua de lluvia para
comunidades vulnerables.''}

\textbf{Importancia:}\\
- Contribuye a resolver \textbf{problemas urgentes}.\\
- Mejora la \textbf{calidad de vida} de poblaciones afectadas.

\begin{center}\rule{0.5\linewidth}{0.5pt}\end{center}

\subsection{\texorpdfstring{\textbf{6. Justificación por Corroboración
de una
Teoría}}{6. Justificación por Corroboración de una Teoría}}\label{justificaciuxf3n-por-corroboraciuxf3n-de-una-teoruxeda}

Se basa en la necesidad de validar o refutar una teoría existente.

\textbf{Ejemplo:}\\
\emph{``Se analizará si los principios de la teoría de redes sociales
pueden aplicarse a la difusión de noticias falsas en plataformas
digitales.''}

\textbf{Importancia:}\\
- Aporta \textbf{evidencia empírica} para teorías existentes.\\
- Puede fortalecer o refutar \textbf{modelos científicos} previos.

\begin{center}\rule{0.5\linewidth}{0.5pt}\end{center}

\subsection{\texorpdfstring{\textbf{7. Justificación por Criterios
Científicos}}{7. Justificación por Criterios Científicos}}\label{justificaciuxf3n-por-criterios-cientuxedficos}

Combina diferentes aspectos de la justificación en un marco
estructurado:

\begin{longtable}[]{@{}
  >{\raggedright\arraybackslash}p{(\linewidth - 2\tabcolsep) * \real{0.2667}}
  >{\raggedright\arraybackslash}p{(\linewidth - 2\tabcolsep) * \real{0.7333}}@{}}
\toprule\noalign{}
\begin{minipage}[b]{\linewidth}\raggedright
\textbf{Criterio}
\end{minipage} & \begin{minipage}[b]{\linewidth}\raggedright
\textbf{Pregunta clave}
\end{minipage} \\
\midrule\noalign{}
\endhead
\bottomrule\noalign{}
\endlastfoot
\textbf{Conveniencia} & ¿Para qué sirve el estudio? \\
\textbf{Relevancia} & ¿Quiénes se beneficiarán y cómo? \\
\textbf{Aplicabilidad} & ¿Resuelve un problema real? \\
\textbf{Valor teórico} & ¿Contribuye a la generación de conocimiento? \\
\end{longtable}

\textbf{Ejemplo:}\\
\emph{``El uso de modelos de inteligencia artificial para predecir la
contaminación del aire en la CDMX ayudará a implementar alertas
tempranas y mejorar políticas ambientales.''}

\textbf{Importancia:}\\
- Evalúa la investigación desde \textbf{múltiples dimensiones}.\\
- Justifica tanto \textbf{aplicaciones prácticas} como
\textbf{contribuciones teóricas}.

\section{\texorpdfstring{\textbf{Conclusión}}{Conclusión}}\label{conclusiuxf3n-1}

\subsection{\texorpdfstring{\textbf{Importancia de una buena
Justificación}}{Importancia de una buena Justificación}}\label{importancia-de-una-buena-justificaciuxf3n}

\begin{itemize}
\tightlist
\item
  Un proyecto sin una \textbf{justificación clara} difícilmente obtendrá
  apoyo académico o financiamiento.\\
\item
  La justificación debe \textbf{estar basada en evidencia y conectar con
  problemas reales}.
\end{itemize}

\subsection{\texorpdfstring{\textbf{Preguntas para validar la
justificación}}{Preguntas para validar la justificación}}\label{preguntas-para-validar-la-justificaciuxf3n}

Antes de finalizar, asegúrate de responder estas preguntas:

\begin{itemize}
\tightlist
\item
  ¿Este estudio es necesario?\\
\item
  ¿Aporta algo nuevo?\\
\item
  ¿Cómo se beneficiarán las personas o instituciones con los
  resultados?\\
\item
  ¿Qué impacto tendrá en la disciplina o en la sociedad?
\end{itemize}

\textbf{Una investigación bien justificada tiene mayores probabilidades
de éxito y relevancia.}

\section{Ejemplo de Justificación por Criterios
Científicos}\label{ejemplo-de-justificaciuxf3n-por-criterios-cientuxedficos}

\subsection{\texorpdfstring{\textbf{Predicción de la Contaminación del
Aire en la CDMX Usando Modelos de Aprendizaje
Automático}}{Predicción de la Contaminación del Aire en la CDMX Usando Modelos de Aprendizaje Automático}}\label{predicciuxf3n-de-la-contaminaciuxf3n-del-aire-en-la-cdmx-usando-modelos-de-aprendizaje-automuxe1tico}

\subsubsection{\texorpdfstring{\textbf{1. Conveniencia} (\emph{¿Para qué
sirve?})}{1. Conveniencia (¿Para qué sirve?)}}\label{conveniencia-para-quuxe9-sirve}

La \textbf{Ciudad de México (CDMX)} es una de las urbes más contaminadas
de América Latina. La exposición prolongada a contaminantes como el
\textbf{PM2.5, PM10, ozono (O₃) y dióxido de nitrógeno (NO₂)} afecta la
salud pública y el medio ambiente. Este proyecto busca desarrollar un
modelo de aprendizaje automático para \textbf{predecir los niveles de
contaminación en la CDMX}, permitiendo alertar a la población y
optimizar estrategias de mitigación.

\begin{center}\rule{0.5\linewidth}{0.5pt}\end{center}

\subsubsection{\texorpdfstring{\textbf{2. Relevancia Social}
(\emph{¿Quiénes se beneficiarán y
cómo?})}{2. Relevancia Social (¿Quiénes se beneficiarán y cómo?)}}\label{relevancia-social-quiuxe9nes-se-beneficiaruxe1n-y-cuxf3mo}

El proyecto impactará a distintos sectores de la sociedad en la CDMX:\\
- \textbf{Gobierno de la CDMX y SEMARNAT}: Mejorará la toma de
decisiones para implementar \textbf{contingencias ambientales y
restricciones vehiculares}.\\
- \textbf{Población en riesgo (niños, adultos mayores y personas con
enfermedades respiratorias)}: Podrá \textbf{anticiparse} a episodios de
contaminación alta y ajustar sus actividades.\\
- \textbf{Sector salud}: Mejorará la planificación hospitalaria y la
prevención de enfermedades respiratorias.\\
- \textbf{Investigadores y académicos}: Dispondrán de una metodología
basada en datos para evaluar políticas ambientales.

\begin{center}\rule{0.5\linewidth}{0.5pt}\end{center}

\subsubsection{\texorpdfstring{\textbf{3. Implicaciones Prácticas}
(\emph{¿Resuelve un problema
real?})}{3. Implicaciones Prácticas (¿Resuelve un problema real?)}}\label{implicaciones-pruxe1cticas-resuelve-un-problema-real}

Actualmente, la \textbf{CDMX enfrenta frecuentes contingencias
ambientales}, afectando la movilidad, la salud y la economía. Este
modelo ayudará a \textbf{predecir la calidad del aire con días de
anticipación}, proporcionando información clave para la
\textbf{implementación de restricciones vehiculares, cierres de
industrias contaminantes y alertas de salud pública}.

\begin{center}\rule{0.5\linewidth}{0.5pt}\end{center}

\subsubsection{\texorpdfstring{\textbf{4. Valor Teórico} (\emph{¿Aporta
nuevos
conocimientos?})}{4. Valor Teórico (¿Aporta nuevos conocimientos?)}}\label{valor-teuxf3rico-aporta-nuevos-conocimientos}

\begin{itemize}
\tightlist
\item
  Se analizarán múltiples \textbf{factores meteorológicos y
  contaminantes} con técnicas avanzadas de Machine Learning.\\
\item
  Se compararán modelos como \textbf{Redes Neuronales, XGBoost y LSTM}
  para identificar el más efectivo en la predicción de la
  contaminación.\\
\item
  Se contribuirá al \textbf{desarrollo de modelos predictivos urbanos}
  que puedan aplicarse a otras ciudades con problemas similares.
\end{itemize}

\section{Más Ejemplos por
Categoría}\label{muxe1s-ejemplos-por-categoruxeda}

\begin{center}\rule{0.5\linewidth}{0.5pt}\end{center}

\subsubsection{Justificación basada en falta de
conocimiento}\label{justificaciuxf3n-basada-en-falta-de-conocimiento}

\begin{itemize}
\tightlist
\item
  \textbf{Ejemplo}: ``No hay estudios previos sobre los efectos del
  cambio climático en la biodiversidad de ecosistemas urbanos en América
  Latina.''
\end{itemize}

\subsubsection{Justificación basada en la importancia del
tema}\label{justificaciuxf3n-basada-en-la-importancia-del-tema}

\begin{itemize}
\tightlist
\item
  \textbf{Ejemplo}: ``El uso de inteligencia artificial en la educación
  puede transformar los métodos de enseñanza, por lo que es necesario
  evaluar su impacto en el aprendizaje.''
\end{itemize}

\subsubsection{Justificación basada en los aportes de los
hallazgos}\label{justificaciuxf3n-basada-en-los-aportes-de-los-hallazgos}

\begin{itemize}
\tightlist
\item
  \textbf{Ejemplo}: ``La implementación de sistemas de captura de
  carbono en industrias altamente contaminantes podría reducir en un
  50\% las emisiones de CO₂ en los próximos diez años, contribuyendo a
  la mitigación del cambio climático.''
\end{itemize}

\begin{center}\rule{0.5\linewidth}{0.5pt}\end{center}

\subsubsection{Justificación basada en un vacío
metodológico}\label{justificaciuxf3n-basada-en-un-vacuxedo-metodoluxf3gico}

\begin{itemize}
\tightlist
\item
  \textbf{Ejemplo}: ``Las herramientas actuales de diagnóstico de
  enfermedades neurodegenerativas no permiten detectar etapas tempranas
  de deterioro cognitivo. Este estudio propone un nuevo enfoque basado
  en inteligencia artificial para mejorar la precisión en la detección
  precoz del Alzheimer.''
\end{itemize}

\subsubsection{Justificación basada en la solución de un
problema}\label{justificaciuxf3n-basada-en-la-soluciuxf3n-de-un-problema}

\begin{itemize}
\tightlist
\item
  \textbf{Ejemplo}: ``La eficiencia en el transporte público sigue
  siendo un reto en ciudades en crecimiento; este estudio busca diseñar
  una estrategia de optimización de rutas.''
\end{itemize}

\subsubsection{Justificación basada en la corroboración de una
teoría}\label{justificaciuxf3n-basada-en-la-corroboraciuxf3n-de-una-teoruxeda}

\begin{itemize}
\tightlist
\item
  \textbf{Ejemplo}: ``Se busca validar la hipótesis de que la
  nanotecnología puede mejorar la eficiencia de las células solares en
  un 30\%.''
\end{itemize}

\section{¿Preguntas o dudas?}\label{preguntas-o-dudas}




\end{document}
